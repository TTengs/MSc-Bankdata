\chapter{Related Work}
\label{chap:related}

\section{Systematic Literature Review}
To explore the existing research of the problem, a systematic literature review (SLR) becomes essential. As mentioned by Gregor and Jones, this method not only enhances the researcher's understanding of the problem area but also provides a foundation to justify the importance of constructing an artifact and its potential impact. In this section, I will identify, evaluate and interpret available research, by employing a rigorous approach to gain insights that will shape the rest of the thesis.


\subsection{Introduction to SLR}
The SLR will be related to the first two research questions:
\begin{itemize}
    \textbf{RQ1:} What are the requirements and considerations that an alternative solution must possess to effectively replace Bankdata's current mainframe infrastructure? \\
    \textbf{RQ2:} What existing solutions do other financial institutions apply who use alternatives to mainframe and how do they compare? \\
\end{itemize}

\subsubsection{Planning the review}
The planning phase will include the definition of the research questions, the search strategy, the inclusion and exclusion criteria, and the development of a protocol.
The search strategy will be defined by the search string, the databases to be used, and the search terms. The inclusion and exclusion criteria will be defined by the type of studies, the publication date, and the language of the articles. The protocol will be developed to ensure that the review is conducted in a consistent and reproducible manner.

\subsubsection{Conducting the review}
The conducting phase will include the search for the articles, the selection of the articles, the data extraction, and the quality assessment. The identification of the articles will be done by searching the databases using the search string. The selection of the articles will be done by applying the inclusion and exclusion criteria to the search results. The data extraction will be done by extracting the relevant information from the selected articles. The quality assessment will be done by evaluating the relation to this study.
Output of this phase will be a list of articles that will be used in the analysis. 

\subsubsection{Reporting the review}
The reporting phase will include the analysis of the articles and the presentation of the results. The analysis of the articles will be done by identifying the themes and patterns in the data. The presentation of the results will be by summarizing the findings and drawing conclusions from each article.
This will be the knowledge base assisting in designing and proposing the artifact to solve the problem.

\subsection{Search Strategy}
The search strings to be used in the SLR are defined by the interview analysis, and takes into account themes and patterns that emerged from the interviews. 

\subsubsection{Search String}
(mainframe OR COBOL OR batch) AND (migrat* OR moderniz* OR alternat*) AND (bank* OR "financial institution*")

\begin{table}[!htb]
    \begin{tabular}{ccccc}
    \hline
    Technology &     & Methods                                                                       &     & Fields of Specification          \\ \hline
    Mainframe  & AND & \begin{tabular}[c]{@{}c@{}}migrat* OR\\ moderniz* OR\\ alternat*\end{tabular} & AND & bank* OR 'financial institution' \\ \hline
    \end{tabular}
\end{table}

\subsubsection{Databases}
The databases to be used in the SLR are \textbf{Scopus}, \textbf{ACM Digital Library}, \textbf{EEE Xplore} and \textbf{Google Scholar}. These databases were presented by Kitchenham and Charters as the most relevant databases for computer science research.
ACM Digital Library is a digital library of the Association for Computing Machinery. IEEE Xplore is a digital library of the Institute of Electrical and Electronics Engineers. Scopus is a bibliographic database containing abstracts and citations for academic journal articles. Google Scholar is a freely accessible web search engine that indexes the full text or metadata of scholarly literature across an array of publishing formats and disciplines.

\subsection{Inclusion and Exclusion Criteria}
The inclusion and exclusion criteria are defined by the type of studies, the publication date, and the language of the articles. 
The type of studies to be included are empirical studies, case studies, and literature reviews. The publication date to be included is from 2010 to 2020. The language of the articles to be included is English.

\section{Identification of the Artifacts of the Classes of Problems}


\section{Proposition of Artifacts to Solve the Problem}

